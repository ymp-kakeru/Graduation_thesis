\subsection{研究背景}


\subsection{先行研究}
\ \ 前年度までに開発されたロボットが\label{six_rober}である.森林環境を想定しており,不整地での踏破能力を高めるためにロッカーボギーサスペンションを採用した六輪ロボットである.
3D LiDARを搭載し,森林環境において誤差15cm以下の高精度の三次元地図の作成に成功した.\cite{arita}\\
\ \ しかしながら,この六輪ロボットには複数の問題点が挙げられた.まず,ロボットのサイズが963[mm] $\times$ 962[mm]と森林環境を想定している割には大きい点.さらに重さが約50[kg]あるために取り扱いが非常に不便である点.また,最大の問題として駆動部の構造上,走行困難となる場面が多々あるという点である.\label{problem}

\subsection{研究目的}
\ \ 前述の問題点を考慮し,本研究ではロボットの小型化および構造の簡略化と改善を目的とした.そのための試みとして,ロボットの構造を
\begin{enumerate}
	\item 制御用センターコンピュータ部
	\item 汎用フレーム
	\item 駆動モジュール
\end{enumerate}
の3要素で構成し,各要素について最適化を行った.さらに,森林環境のみならず屋内や平野といった様々な環境に視野を広げるために,2輪・4輪・6輪と組み換え可能なモジュール構造を採用した.