\par 製作したロボットについて,駆動モジュール数を変更することで計測環境に適したロボットとなるか,および高精度の三次元地図の生成を行うことができるかを確認するために検証実験を行った.
\subsection{実験内容}
\par 実験は\ref{fields}に示す3つの環境で行った.それぞれ
\begin{itemize}
	\item 室内(平坦な環境)
	\item 砂利道(起伏のある環境)
	\item 森林(不安定かつ障害物のある環境)
\end{itemize}
である.これらに対して,\ref{frame}に示した汎用フレームに2・4・6つの駆動モジュールを組込んだ3種のロボットを用いて比較のための3次元環境計測実験を行った.
また,それぞれは小回りの利く独立二輪機構,走破性と安定性を重視した四輪機構,不整地に対する走破性に重きをおいた六輪ロッカーボギー機構を有する.

\subsection{実験手順}
\par 各実験環境において,各種ロボットを遠隔操作によって走行させ,3次元地図を作成した.その際に,走行・計測完了までのタイム,地図の精度,(検討中)について比較を行う.地図の精度についてはその要因となりうるロボット本体の揺れについても,3DLiDAR(\ref{LiDAR})のIMU機能を用いて検証を行う.
