\subsection{センターコンピュータ}
\par ロボットの制御用センターコンピュータの構成について以下にまとめる.
\subsubsection{モータドライバ}

\par モータドライバはCytron Technologies製 2ch DCブラシモータドライバを使用した.本製品はPWM信号を入力することにより速度制御を行うことができる.\\
\par ロジック電圧は5[V]であり,後述のモータコントローラを介してPCから給電される.

\subsubsection{モータコントローラ}
\par エンコーダパルスカウンタおよびモータコントローラには iXs Research製 超小型USB接続4chモータコントローラを使用した.本製品は最大3k組のエンコーダ・DCモータの制御が可能である.また,入力には2相エンコーダ入力信号(4000pps未満),出力はPWM信号・CW/CCWが利用可能である.さらに,USBでPCから直接制御することが可能かつ,内部にPID制御式(\ref{imcs_eq})を持っているためにプログラム上でゲインの設定を行うだけで用意に制御が可能である.
制御出力 u は次式から 1[ms]毎に計算される.

\begin{eqnarray}
	u=A-\frac{K_P}{K_{P_x}}(v_d - v)%-\frac{K_D}{K_{D_x}}(\dot v_d - \dot x)
	-\frac{K_I}{K_{I_v}}\sum_0^t (v_d-v)
\label{imcs_eq}
\end{eqnarray}

このとき,Aはオフセットであり,$K_P・K_I$はそれぞれ比例・積分ゲインである.また,$v_d$は目標速度,$v$は現在値である.\\
\par ただし,本モータコントローラは本来位置制御に特化したものであるので現在値を得る際には,
\begin{equation}
	v = \frac{x_i - x_{i-1}}{t}
\end{equation}
から求める.このとき$x_i$は現在位置, $x_{i-1}$は1ステップ前の位置であり,$t$は1ステップの時間である.

\subsubsection{電源}
石川鉄鋼所にきかなければー
%--------------------------------------------------------------------------------------------------
\subsection{汎用フレーム}







%--------------------------------------------------------------------------------------------------
\subsection{駆動モジュール}
\subsubsection{DCモータ}
\subsubsection{ロータリエンコーダ}
\subsubsection{モジュール構成}



%--------------------------------------------------------------------------------------------------
\subsection{3DLiDAR}

